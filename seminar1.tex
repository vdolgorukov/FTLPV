\section{Вычислительная сложность, DEL и Theory of Mind (01.04.2020)}

\begin{description}
\item[Обсуждаемая работа]\autocite{VandePol2018}
\item[См. также:] \autocite{Abramov2009d,Rubio-Fernandez2013a,Szymanik2018TractabilityMind}
\end{description}


Ключевые идеи:
\begin{itemize}
\item трудность задачи false-belief task обусловлена не модальной глубиной установок (<<я думаю, что он думает...>>), а целым списком параметров 
\item  false-belief task можно моделировать средствами DEL
\item  задачи из DEL имеют разную вычислительную сложность
\item НО: нужна не стандартная теория вычислительной сложности, а параметрическая
\item P-тезис, утверждающий, что реальные когнитивные задачи могут относится только к классу P (класс задач, решаемых за полиномиальное время) нужно ослабить
\item некоторые решаемы на практике задачи могут относится к более сложному классу NP, при некотором параметре (класс para-NP) и даже более сложному классу (параметризованный аналог PSPACE)
\end{itemize}

\subsection{Вычислительная сложность}
Задача: найти загаданное число от 1 до 1000
Алгоритм 1: это 1? это 2?
Алгоритм 2: Это число меньше 500? Это число меньше 250?

\subsection{Полиномиальная vs. экспоненциальная сложность}
$O(n^2)$ (полиномиальная) vs. $O(2^n)$ (экспоненциальная)

Сравнение скоростей роста функции:
\begin{equation}
    \log  n < \sqrt{n} < n < n  \log  n < n^2 < 2^n
\end{equation}


\subsection{Некоторые классы вычислительной сложности}

\begin{center}
$P \subset NP \subset PSPACE \subset EXPTIME$\footnote{
Cтрогость включения  является гипотезой.}	
\end{center}

\subsection{Hard vs. Complete}
$NP-complete =  NP \cap NP-hard$

\subsection{Сложность некоторых логических задач}

\begin{description}
\item[$SAT$:] Является ли формула выполнимой?
\item[$VALIDITY$:] Является ли формула общезначимой?
\item[$MODEL-CHECKING$:] Выполняется ли данная формула в данной модели?	
\end{description}


\subsection{ToM и False Belief Task}
Считается, что FBT не проходят дети  до 4 лет, но наш гнс Барт Гертс (и Паола Рубио-Фернандез) показали, что можно и раньше см: \autocite{Rubio-Fernandez2013a}
\subsection{P-тезис}

\subsection{Вопросы}
\begin{itemize}
    \item Существуют ли более богатые, чем хорновский фрагменты логики высказываний, обладающие полиномиальной сложностью?
    \item Почему класс EXP называется экспоненциальным, если функции описывающие рост сложности показательные?
\end{itemize}

\subsection{Полезные ссылки}
\begin{itemize}
    \item \url{https://lectoriy.mipt.ru/course/Maths-ComputationalComplexity-14L}
\end{itemize}