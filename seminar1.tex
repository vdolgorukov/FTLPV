

\begin{description}

\section{DEL, вычислительная сложность и False-Belief Task (01.04.2020)}
\item[Обсуждаемая работа]\autocite{VandePol2018}
\item[См. также:] \autocite{Abramov2009d,Rubio-Fernandez2013a,Szymanik2018TractabilityMind}
\end{description}


Ключевые идеи:
\begin{itemize}
\item трудность задачи false-belief task обусловлена не модальной глубиной установок (<<я думаю, что он думает...>>), а целым списком параметров 
\item  false-belief task можно моделировать средствами DEL
\item  задачи из DEL имеют разную вычислительную сложность
\item НО: нужна не стандартная теория вычислительной сложности, а параметрическая
\item P-тезис, утверждающий, что реальные когнитивные задачи могут относится только к классу P (класс задач, решаемых за полиномиальное время) нужно ослабить
\item некоторые решаемы на практике задачи могут относится к более сложному классу NP, при некотором параметре (класс para-NP) и даже более сложному классу (параметризованный аналог PSPACE)
\end{itemize}

\subsection{исследования Theory of mind при шизофрении и аутизме}
    Возникший в философии в конце 20 века концепт «Theory of Mind» (ToM) стал значимым в психологии 21 века. Обычно термин не переводят на русский язык, но всё же в русскоязычной литературе можно встретить различные варианты перевода: «индивидуальная теория психики» (Величковский, 2006), «модель психического» (Сергиенко, Лебедева, Прусакова, 2009), или «внутренняя модель сознания другого» (Лоскутова, 2009).
ТоМ можно связать с привычным концептом эмпатии. В психологии ТоМ понимается как отражение содержания сознания другого человека, которое, однако, является ситуативным, изменчивым и субъективным. Сложность подобного концепта для исследования заключается в том, что ТоМ является феноменом индивидуального сознания и может не иметь выражения в поведении, а выводы о содержании сознания другого, к которым приходит субъект, строятся преимущественно бессознательно. Тем не менее, достаточно быстро эмпирические исследования на основе предложенной концептуальной модели смогли зафиксировать нарушения ТоМ в рамках различных нозологических единиц в психиатрии. Клинические психологи воодушевились новым концептом и связали с ним большие надежды.
Впервые гипотеза о нарушении ТоМ при раннем детском аутизме была высказана в работе Саймона Барона-Коэна, Алана Лесли и Уты Фриз (Baron-Cohen, Leslie, Frith, 1985), в которой было показано, что аутистичные дети хуже справляются с рядом задач на социальное взаимодействие. Это открытие позволило приблизиться к комплексному описанию центрального дефекта при аутизме. Позже эти же исследователи поставили известную ныне границу формирования ТоМ в норме на уровне 4 лет. Поскольку работа шла с маленькими детьми, а также в силу сложности самого концепта, клинические психологи вынуждены были проявить изобретательность при разработке стимульного материала, разыгрывая различные вариации известной ныне задачи на ложные убеждения (false belief task) . Такое объяснение стало крайне продуктивным для понимания существующего при некоторых формах высокофункционального аутизма расхождения между социальной некомпетентностью аутистичных персон и сохранными способностями к решению задач, лишённых социального компонента. Однако в  большинстве случаев связь нарушений ТоМ с нарушения интеллекта является прямой: эти виды дефектов поддерживают и усугубляют друг друга. Одна из основных гипотез связывает нарушение ТоМ с нарушением работы исполнительных функций (функций контроля за исполнением деятельности) (Ozonoff, et al., 1991, Baron-Cohen, Swettenham, 1997, Rajendran, Mitchell, 2007, Ballroom, Foyer, 2011 и др.) .
Огромные надежды с ТоМ связали также и исследователи шизофрении: количество ссылок на исследования дефицита ТоМ при шизофрении в словарных статьях может достигать нескольких тысяч. Одним из первых обосновать дефицит ТоМ в качестве центрального при шизофрении пытался Кристофер Фриз (Frith, 1992). Он отмечал, что данный дефицит имеется до манифестации заболевания, фактически указывая на него как на один из предикторов шизофрении. Фриз показывает, как можно объяснить клинические проявления шизофрении на основе дефицита ТоМ и связанного с ним когнитивного дефицита. Он понимал дефицит ТоМ как обусловленный биологически, тем самым возвращая психиатрам надежды на открытие биологического механизма болезни. Эмпирические исследования нашли подтверждение многим идеям, высказанным Фризом, но некоторые идеи не подтвердились на достаточном уровне значимости (например, была установлена значительная роль нарушения ТоМ для негативной симптоматики, тогда как остальные синдромы явного нарушения ТоМ не обнаружили, хотя и всегда результаты были ниже нормы. При параноидной же шизофрении дефицит ТоМ значительно уменьшался после лечения (Corcoran, Cahill, Frith, 1997). Наилучшие же результаты были продемонстрированы пациентами с психическими автоматизмами (Frith, Corcoran, 1996)).
Сам концепт в ходе исследований неизбежно эволюционирует. Если Фриз понимал ТоМ как некую целостную структуру («модульная» трактовка), в которой можно выделить различные стороны, то для современных исследователей ТоМ скорее представляет собой термин, за которым скрывается множество хоть и связанных между собой, но всё же относительно независимых функций («молекулярная» трактовка) и изучается связь этих функций с различными проявлениями нейродефицита. 
Впрочем, не следует поддаваться иллюзии, будто дефицит ТоМ при шизофрении является общим местом для всех исследователей. Например, Абу-Акель (Abu-Akel, 1999) считал, что больные шизофренией обладают гипертрофированной способностью с ТоМ.
Итак, значимость концепта ТоМ в клинической психологии очень сложно переоценить. Как часто это бывает, многие эмпирические исследования ТоМ демонстрируют противоречивые результаты. Чтобы как-то подытожить данный обзор, я постараюсь перечислить основные направления разработки ТоМ в современной клинической психологии:
1.	Какая трактовка, молекулярная или модульная, более правдоподобна
2.	Структура ТоМ;
3.	Является ли нарушение ТоМ устойчивым;
4.	Как ТоМ связано с другими видами когнитивного дефицита;
5.	Связь нарушений ТоМ и нейродефицита;
6.	Роль нарушений ТоМ в структуре различных симптомокомплексов;
7.	Источник формирования ТоМ: вклад социальных, психологических, биологических факторов . 


\subsection{Вычислительная сложность}
Задача: найти загаданное число от 1 до 1000
Алгоритм 1: это 1? это 2?
Алгоритм 2: Это число меньше 500? Это число меньше 250?

\subsection{Полиномиальная vs. экспоненциальная сложность}
$O(n^2)$ (полиномиальная) vs. $O(2^n)$ (экспоненциальная)

Сравнение скоростей роста функции:
\begin{equation}
    \log  n < \sqrt{n} < n < n  \log  n < n^2 < 2^n
\end{equation}


\subsection{Некоторые классы вычислительной сложности}

\begin{center}
$P \subset NP \subset PSPACE \subset EXPTIME$\footnote{
Cтрогость включения  является гипотезой.}
\end{center}

\subsection{Hard vs. Complete}
$NP-complete =  NP \cap NP-hard$

Пусть $A$ и $B$ суть два языка. Тогда $A$ сводится по Карпу к $B$, если существует всюду определённая функция $f$, вычислимая за полиномиальное время, такая что $x\in A \Longleftrightarrow f(x) \in B$. 

Язык $B$ является NP-трудным, если любое $A \in$ NP сводится по Карпу к $B$. Язык $B$ является NP-полным, если он NP-трудный и лежит в NP.


\subsection{Сложность некоторых логических задач}

\begin{description}
\item[$SAT$:] Является ли формула выполнимой?
\item[$VALIDITY$:] Является ли формула общезначимой?
\item[$MODEL-CHECKING$:] Выполняется ли данная формула в данной модели?	
\end{description}


\subsection{ToM и False Belief Task}
Считается, что FBT не проходят дети  до 4 лет, но наш гнс Барт Гертс (и Паола Рубио-Фернандез) показали, что можно и раньше см: \autocite{Rubio-Fernandez2013a}

\subsection {False-Belief task vs. Belief-task}

Для ряда исследователей, принадлежащих традиции аналитической феноменологии, (см. к прим. Zahavi, 2014) прояснение того, как агент в раннем возрасте приписывает другому агенту наличие сознания, требует не только и не столько обращения к false-belief tasks, сколько к belief-tasks. False-belief tasks обогащают понимание природы познания чужих убеждений, желаний и переживаний, но не вскрывают до конца основания этого познания. 
Дети в 4-летнем (возможно, и более раннем) возрасте способны оценивать содержание убеждений. Но, прежде, важно выявить то, как образуется знание о наличии убеждения. Так, дети значительно более младшего возраста (в том числе, в возрасте нескольких месяцев), которые не обладают знанием о содержании убеждений, желаний и намерений другого агента, тем не менее обладают (пре-рефлексивной) способностью приписывать наличие каких-либо намерений у того, с кем у них есть какой-либо контакт (Csibra 2010). Иначе говоря, ребенок на первых стадиях развития способен к распознанию каких-либо намерений  в его адрес. Прежде чем у ребенка складывается понимание содержания убеждений и намерений, он уже способен отличить другого агента от неодушевленного предмета (Reid and Striano, 2007): то есть на этом уровне развития уже наличествует некоторая базовая интерсубъективность. Младенец испытывает переживание “обмена опытом” с другими, хотя и на крайне ограниченном уровне.
Так, для того, чтобы в дальнейшем ребенок мог оценить содержание тех убеждений, которые есть, к примеру, у его матери, необходимо, чтобы на более ранних стадиях он мог распознать разницу между матерью и другим человек, или между матерью и столом. Такая базовая способность — необходимое условие для развития коммуникативных навыков и также необходимая составляющая для более комплексной ToM. 


\subsection{P-тезис}
Когнитивные задачи, которые наша психика способна решать, относятся к классу $P$.

\subsection{Вопросы}
\begin{itemize}
    \item Существуют ли более богатые, чем хорновский фрагменты логики высказываний, обладающие полиномиальной сложностью?
    \item Почему класс EXP называется экспоненциальным, если функции описывающие рост сложности показательные?
\end{itemize}

\subsection{Полезные ссылки}
\begin{itemize}
    \item \url{https://lectoriy.mipt.ru/course/Maths-ComputationalComplexity-14L}
\end{itemize}