\documentclass[13pt]{paper}
\usepackage[utf8]{inputenc} 
\usepackage[T2A]{fontenc} 
\usepackage[russian, english]{babel}
\usepackage{amsmath,amssymb}
\usepackage{url}
\usepackage{csquotes}

\title{Материалы семинара <<From the Logical Point of View>>}
\subtitle{Краткие заметки посреди чумы}
\date{логика посреди чумы}
\author{}
 

 %%%%%%%%%%%%%%%%%%%%%%%%%%
%Настройки библиографии
%%%%%%%%%%%%%%%%%%%%%%%%%%

\usepackage[
style=gost-authoryear-min,
%authoryear,
%citestyle=authoryear,
%bibstyle=gost-authoryear,
mergedate=maximum, %перемещает дату в начало
movenames=false, % не перемещает имена авторов после /
maxbibnames=5, % максимальное число авторов
%firstinits=true,
isbn=false,
url=false,
%doi=false,
%dashed = false,
autolang=other,
backend=biber%
]{biblatex}


%\let\emph\textit %убирает подчеркивание в библиографии

%%% меняет поле language на langid
\DeclareSourcemap{ %language-> langid
 \maps{
    \map{
     \step[fieldsource=language, fieldset=langid, origfieldval]
     \step[fieldset=language, null]  }
    \map{
     \step[fieldsource=language, fieldset=keywords, origfieldval, final]
     \step[fieldset=language, null]  }   
     }}

\DeclareSourcemap{
  \maps[datatype=bibtex]{
    \map{
      \step[fieldset=langid, fieldvalue={english}]
    }
  }
}

%%% Сначала русские источники, потом - иностранные
\DeclareSourcemap{
\maps[datatype=bibtex]{
\map{
\step[fieldsource=langid, match=russian, final]
\step[fieldset=presort, fieldvalue={a}]
}
\map{
\step[fieldsource=langid, notmatch=russian, final]
\step[fieldset=presort, fieldvalue={z}]
}
}
}


%%%%%Настройки ГОСТА
\renewcommand{\newblockpunct}{% убирает — 
\addperiod\space\bibsentence}%block punct.,\bibsentence is for vol,etc.


\DefineBibliographyStrings{english}{%
  pages = {P\adddot},}
\DefineBibliographyStrings{english}{%
  number = {No\adddot},}
\DefineBibliographyStrings{russian}{%
  pages = {c\adddot},}  

%убирает пробел перед : и ;
\renewcommand*{\addcolondelim}{\addcolon\space}
\renewcommand*{\addsemicolondelim}{\addsemicolon\space}

%%Меняем [] на ()%%%%%%%%%%
\renewcommand{\mkbibbrackets}[1]{\mkbibparens{#1}}

%\AtEveryBibitem{\clearfield{month}}
%\AtEveryBibitem{\clearfield{day}}
%\AtEveryBibitem{\clearfield{file}}
%\AtEveryBibitem{\clearfield{language}}
\AtEveryBibitem{\clearfield{pagetotal}}


%%убирает скобки из [et al]
%\newcommand*{\mkandothers}{\mkbibemph}
\renewbibmacro*{name:andothers}{%
  \ifboolexpr{
    test {\ifnumequal{\value{listcount}}{\value{liststop}}}
    and
    test \ifmorenames
  }
    {\ifnumgreater{\value{liststop}}{1}
       {\finalandcomma}
       {}%
     \andothersdelim\bibstring[\mkandothers]{andothers}}
    {}}