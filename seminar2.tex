\section{DEL и две системы (08.04.2020)}
обсуждаемая статья \autocite{Solaki2019TheThinking}

\subsection{Две системы}

\begin{itemize}
	\item Система-1: быстрая
	\item Система-2: медленная
\end{itemize}

\begin{quote}
The main function of System 1 is to maintain and update a model of your personal world, which represents what is normal in it. [...] System 1 excels at constructing the best possible story that incorporates ideas currently activated, but it does not (cannot) allow for information it does not have. (Kahneman 2011, p. 71 and p. 85)	
\end{quote}

\begin{quote}
I describe System 1 as effortlessly originating impressions and feelings that are the main sources of the explicit beliefs and deliberate choices of System 2. The automatic operations of System 1 generate surprisingly complex patterns of ideas, but only the slower System 2 can construct thoughts in an orderly series of steps. (Kahneman 2011, p. 21)	
\end{quote}

\subsection{Succession (def. 4.4)}

Из мира $w_1$ достижим мир $w_w$ при помощи правила, если 1) в мире 1 истинны посылки правила 2) в мире 1 нет явного противоречия с заключнием правила 3) в мире 2 истинны все те же формул, что и в 1 + заключение правила



$\Box \varphi$ означает, что $\varphi$ истинно в мире 0

For the possible world w1 , we list only the propositional atoms it satisfies, since all the rest can be computed recursively

\subsection{Вопросы}

\begin{itemize}
    \item Зачем тут ординалы?
    \item Что конкретно подразумевается под "connectedness"?
    \item нет ли ошибки на рис 1. внзу справа стрелочк идет от мира 1 к миру 2?
\end{itemize}


